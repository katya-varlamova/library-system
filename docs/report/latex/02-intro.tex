\chapter*{Введение}
\addcontentsline{toc}{chapter}{ВВЕДЕНИЕ}

Библиотека -- это учреждение, собирающее и осуществляющее хранение книг и иных печатных изданий для общественного пользования. 


Электронная библиотечная система (ЭБС) является средством управления хранения книг и процессом получения доступа к ним. Такие системы значительно упрощают процесс учёта книг из библиотечного фонда. В настоящее время ЭБС используются многими организациями для обеспечения своих сотрудников необходимой литературой, университетами и школами -- для организации учебного процесса, а также городскими библиотеками. 

Цель работы -- разработать базу данных электронной библиотечной системы, которая позволит читателям получать информацию о доступных книгах в разных библиотеках, библиотекарям -- выдавать и принимать книги, а администраторам библиотечной системы -- редактировать информацию о книгах и библиотеках.

Для достижения поставленной цели требуется решить следующие задачи:

\begin{itemize}
    \item проанализировать существующие решения задачи, выявить их недостатки;
    \item сформулировать требования к разрабатываемому приложению с учётом выявленных недостатоков у аналогов;
    \item проанализировать варианты представления данных и выбрать подходящий вариант для решения задачи;
    \item спроектировать базу данных, описать ее сущности и связи;
    \item разработать архитектуру приложения;
    \item реализовать программное обеспечение на основе разработанной структуры приложения;
    \item провести исследование производительности в зависимости от наличия индексов.
\end{itemize}
